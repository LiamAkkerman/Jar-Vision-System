\documentclass[11pt]{article}

\usepackage[utf8]{inputenc}
\usepackage{graphics}
\usepackage{graphicx}
\usepackage[margin=3cm]{geometry}
\usepackage{placeins}

\usepackage{titlesec}
\usepackage[titles]{tocloft}
\linespread{1.25}
\setlength{\parindent}{0pt}
\setlength{\parskip}{1em}
\titlespacing{\section}{0pt}{\parskip}{-\parskip}
\titlespacing{\subsection}{0pt}{\parskip}{-\parskip}
\titlespacing{\subsubsection}{0pt}{\parskip}{-\parskip} %TODO adjust each of these once more text has been filled
\setlength{\cftbeforesecskip}{3pt}
\setlength{\cftbeforesubsecskip}{0pt}

\usepackage{hyperref} 
\hypersetup{
    colorlinks   = true,
    urlcolor     = blue,
    linkcolor    = black
}

\title{{\large MSE491 - Application of Machine Learning in Mechatronic Systems} \\ Project Report: \\ Vessel Targeting System for Automated Gantry Robot}
\author{Group 25 \\ Liam Akkerman - 301286906 \\ Aidan Hunter - 301279938}

\begin{document}
    \maketitle
    \vfill
	\setcounter{tocdepth}{2} % only include down to subsections
    \tableofcontents % could move to new page if too long
    \FloatBarrier 
    \newpage

    \FloatBarrier
    \section{Introduction}
        \subsection{Problem Description}
            % jars need to move from loose in a tray to a conveyor belt

        \subsection{Project Limitations}
            % needs to be on a rpi0. output meaningful for a targeting system.

        \subsection{Design Philosophy}
            % ehhh I dunno. big data = good data.

    \FloatBarrier
    \section{Methodology}
        \subsection{Data Collection}
            \subsubsection{Camera Frame}
                % physical frame the camera and pi are mounted on

            \subsubsection{Database}
                % storing all the data points

            \subsubsection{Data Point Labelling}
                % taking photos and the GUI program

        \subsection{Machine Learning Model}
            % the Keras and other ML stuff
            \subsubsection{Data Augmentation} % maybe this should be in data collection
                % bolstering the data set with image transformations 

            \subsubsection{Keras Model Layers}
                % describing each layer in the model

        \subsection{Project Implementation and Integration}
            % how we got it to run on our hardware and how it will be integrated with the next subsystem
            \subsubsection{Tensor Flow on the Raspberry Pi}
                % how it got installed on armv6 and the tflite model exporting/importing

            \subsubsection{Raspberry Pi Program}
                % the program which executes on the pi. uses a pi camera to take and store photos for training. invokes the model when in use. full command line control possible.

            \subsubsection{Deriving Meaningful Data from the Model}
                % I don't know this yet

    \FloatBarrier
    \section{Results}
        \subsection{Example Results}
            % pictures visualizing the output of sample data

        \subsection{Results Metrics}
            % measurements quantifying the results
    
    \FloatBarrier
    \section{Discussions}
        % lots of graphs

    \FloatBarrier
    \section{Conclusion}

    \appendix
	\section{List of Figures}
		\makeatletter
		\@starttoc{lof}% Print List of Figures
		\makeatother

	\section{List of Tables}
		\makeatletter
		\@starttoc{lot}% Print List of Tables
		\makeatother
    
    \section{References}

\end{document}
